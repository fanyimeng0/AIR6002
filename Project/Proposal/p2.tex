\documentclass[a4paper,11pt]{scrartcl}

\usepackage{graphicx}
\usepackage[utf8]{inputenc} %-- pour utiliser des accents en français, ou autres
\usepackage{amsmath,amssymb,amsthm} 
\usepackage[round]{natbib}
\usepackage{url}
\usepackage{xspace}
\usepackage[left=20mm,top=20mm]{geometry}
\usepackage{algorithmic}
\usepackage{subcaption}
\usepackage{mathpazo}
\usepackage{booktabs}
\usepackage{hyperref}


\newcommand{\ie}{i.e.}
\newcommand{\eg}{e.g.}
\newcommand{\reffig}[1]{Figure~\ref{#1}}
\newcommand{\refsec}[1]{Section~\ref{#1}}

\setcapindent{1em} %-- for captions of Figures

\renewcommand{\algorithmicrequire}{\textbf{Input:}}
\renewcommand{\algorithmicensure}{\textbf{Output:}}


\title{Mid-term Proposal}
\author{Fanyi Meng \\ \url{fanyimeng@link.cuhk.edu.cn}}
\date{March 1, 2024}


\begin{document}

\maketitle


%%%
%
\section{Introduction}
An introduction in which the relevance of the project and its place in the context of the end goals of the MSc Geomatics is described, along with a clearly-defined problem statement.

%%%
%
\section{Related work}
A related work section in which the relevant literature is presented and linked to the project.

\citet{Delaunay34} cites something as a noun, and it's also possible to put the references between parentheses at the end of at sentence~\citep{Voronoi08}.

%%%
%
\section{Research questions}
The research questions (and sub-questions) are clearly defined, along with the scope (\ie\ what you will not be doing).


%%%
%
\section{Methodology}
Overview of the methodology to be used.

%%%
%
\section{Time planning}
Having a Gantt chart is probably a better idea then just a list.

%%%
%
\section{Tools and datasets used}
Since specific data and tools have to be used, it’s good to present these concretely, so that the mentors know that you have a grasp of all aspects of the project.


\bibliographystyle{abbrvnat}
\bibliography{myreferences}
\end{document}