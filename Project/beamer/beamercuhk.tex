\documentclass{beamer}
\usepackage{amsfonts,amsmath,oldgerm,mathrsfs,booktabs}
\usetheme{sintef}
\usepackage{xeCJK}

\newcommand{\testcolor}[1]{\colorbox{#1}{\textcolor{#1}{test}}~\texttt{#1}}

\usefonttheme[onlymath]{serif}

\titlebackground*{assets/background}

\newcommand{\hrefcol}[2]{\textcolor{cyan}{\href{#1}{#2}}}

\title{Zero-Shot Learning of Image Classification
through Text-to-Image Generative Model}

\course{Advanced Machine Learning(AIR 6002)}
\author{Fanyi Meng}

\IDnumber{223015127}
\date{\today}

\begin{document}
\maketitle

% \begin{frame}

% This template is a based on \hrefcol{https://www.overleaf.com/latex/templates/sintef-presentation/jhbhdffczpnx}{SINTEF Presentation} from \hrefcol{mailto:federico.zenith@sintef.no}{Federico Zenith} and its derivation \hrefcol{https://github.com/TOB-KNPOB/Beamer-LaTeX-Themes}{Beamer-LaTeX-Themes} from Liu Qilong

% \vspace{\baselineskip}

% CUHK style adaptation contributed by \hrefcol{https://richardfury.github.io/}{Rui HU}

% \vspace{\baselineskip}

% In the following you find a brief introduction on how to use \LaTeX\ and the beamer package to prepare slides, based on the one written by \hrefcol{mailto:federico.zenith@sintef.no}{Federico Zenith} for \hrefcol{https://www.overleaf.com/latex/templates/sintef-presentation/jhbhdffczpnx}{SINTEF Presentation}

% % This template is released under \hrefcol{https://creativecommons.org/licenses/by-nc/4.0/legalcode}{Creative Commons CC BY 4.0} license

% \end{frame}

\section{Introduction}
\footlinecolor{maincolor}

\begin{frame}[fragile]{Zero-shot Learning}
\framesubtitle{------}
\begin{itemize}
\item A typical slide has bulleted lists
\item These can be uncovered in sequence
\end{itemize}
\begin{block}{Code for a Page with an Itemised List}<+->
\begin{verbatim}
\begin{frame}{Writing a Simple Slide}
  \framesubtitle{It's really easy!}
  \begin{itemize}[<+->]
    \item A typical slide has bulleted lists
    \item These can be uncovered in sequence
  \end{itemize}\end{frame}
\end{verbatim}
\end{block}
\end{frame}

\section{Personalization}

\footlinecolor{sintefyellow}
\begin{frame}[fragile]{Changing Slide Style}
\begin{itemize}
\item You can select the white or \textit{maincolor} \textbf{slide style} \emph{in the 
preamble} with \verb|\themecolor{white}| (default) or \verb|\themecolor{main}|
      \begin{itemize}
      \item You should \emph{not} change these within the document: Beamer does 
      not like it
      \item If you \emph{really} must, you may have to add 
      \verb|\usebeamercolor[fg]{normal text}| in the slide
      \end{itemize}
\item You can change the \textbf{footline colour} with 
\verb|\footlinecolor{color}|
      \begin{itemize}
      \item Place the command \emph{before} a new \verb|frame|
      \item There are four ``official'' colors: 
      \testcolor{maincolor}, \testcolor{sintefyellow}, 
      \testcolor{sintefgreen}, \testcolor{sintefdarkgreen}
      \item Default is no footline; you can restore it with 
      \verb|\footlinecolor{}|
      \item Others may work, but no guarantees!
      \item Should \emph{not} be used with the \verb|maincolor| theme!
      \end{itemize}
\end{itemize}
\end{frame}

\begin{frame}[fragile]{Blocks}
\begin{columns}
\begin{column}{0.3\textwidth}
\begin{block}{Standard Blocks}
These have a color coordinated with the footline (and grey in the blue theme)
\begin{verbatim}
\begin{block}{title}
content...
\end{block}
\end{verbatim}
\end{block}
\end{column}
\begin{column}{0.7\textwidth}
\begin{colorblock}[black]{sinteflightgreen}{Colour Blocks}
Similar to the ones on the left, but you pick the colour. Text will be white by 
default, but you may set it with an optional argument.
\small
\begin{verbatim}
\begin{colorblock}[black]{sinteflightgreen}{title}
content...
\end{colorblock}
\end{verbatim}
\end{colorblock}
The ``official'' colours of colour blocks are: \testcolor{sinteflilla}, 
\testcolor{maincolor}, \testcolor{sintefdarkgreen}, and 
\testcolor{sintefyellow}.
\end{column}
\end{columns}
\end{frame}

\footlinecolor{}
\begin{frame}[fragile]{Using Colours}
\begin{itemize}[<alert@2>]
  \item You can use colours with the
        \verb|\textcolor{<color name>}{text}| command
  \item The colours are defined in the \texttt{sintefcolor} package:
  \begin{itemize}
  \item Primary colours: \testcolor{maincolor} and its sidekick 
  \testcolor{sintefgrey}
  \item Three shades of green: \testcolor{sinteflightgreen}, 
  \testcolor{sintefgreen}, \testcolor{sintefdarkgreen}
  \item Additional colours: \testcolor{sintefyellow}, \testcolor{sintefred}, 
        \testcolor{sinteflilla}
        \begin{itemize}
        \item These may be shaded---see the \verb|sintefcolor| documentation or 
        the \hrefcol{https://sintef.sharepoint.com/sites/stottetjenester/%
        kommunikasjon/grafisk-profil-new/Sider/default.aspx}{SINTEF profile 
        manual}
        \end{itemize}
  \end{itemize}
  \item Do \emph{not} abuse colours: \verb|\emph{}| is usually enough
  \item Use \verb|\alert{}| to bring the \alert<2->{focus} somewhere
  \item<2- | alert@2> If you highlight too much, you don't highlight at all!
\end{itemize}
\end{frame}

\begin{frame}[fragile]{Adding images}
\begin{columns}
\begin{column}{0.7\textwidth}
Adding images works like in normal \LaTeX:
\begin{block}{Code for Adding Images}
\begin{verbatim}
\usepackage{graphicx}
% ...
\includegraphics[width=\textwidth]
{assets/logo_RGB}
\end{verbatim}
\end{block}
\end{column}
\begin{column}{0.3\textwidth}
\includegraphics[width=\textwidth]
{assets/logo_RGB}
\end{column}
\end{columns}
\end{frame}

\begin{frame}[fragile]{Splitting in Columns}
Splitting the page is easy and common;
typically, one side has a picture and the other text:
\begin{columns}
\begin{column}{0.6\textwidth}
This is the first column
\end{column}
\begin{column}{0.3\textwidth}
And this the second
\end{column}
\end{columns}
\begin{block}{Column Code}
\begin{verbatim}
\begin{columns}
    \begin{column}{0.6\textwidth}
        This is the first column
    \end{column}
    \begin{column}{0.3\textwidth}
        And this the second
    \end{column}
    % There could be more!
\end{columns}
\end{verbatim}
\end{block}
\end{frame}

\begin{chapter}[assets/background_negative]{}{Special Slides}
\begin{itemize}
\item Chapter slides
\item Side-picture slides
\end{itemize}
\end{chapter}

\footlinecolor{sintefred}
\begin{frame}[fragile]{Chapter slides}
\begin{itemize}
\item Similar to \verb|frame|s, but with a few more options
\item Opened with \verb|\begin{chapter}[<image>]{<color>}{<title>}|
\item Image is optional, colour and title are mandatory
\item There are seven ``official'' colours: \testcolor{maincolor}, 
\testcolor{sintefdarkgreen}, \testcolor{sintefgreen}, 
\testcolor{sinteflightgreen}, \testcolor{sintefred}, \testcolor{sintefyellow}, 
\testcolor{sinteflilla}.
      \begin{itemize}
      \item Strangely enough, these are \emph{more} than the official colours 
      for the footline.
      \item It may still be a nice touch to change the footline of following 
      slides to the same color of a chapter slide. Your choice.
      \end{itemize}
\item Otherwise, \verb|chapter| behaves just like \verb|frame|.
\end{itemize}
\end{frame}

\begin{sidepic}{assets/background_alternative}{Side-Picture Slides}
\begin{itemize}
\item Opened with \texttt{$\backslash$begin\{sidepic\}\{<image>\}\{<title>\}}
\item Otherwise, \texttt{sidepic} works just like \texttt{frame}
\end{itemize}
\end{sidepic}

\footlinecolor{}
\begin{frame}
\frametitle{Fonts}
\begin{itemize}
\item The paramount task of fonts is being readable
\item There are good ones...
  \begin{itemize}
  \item {\textrm{Use serif fonts only with high-definition projectors}}
  \item {\textsf{Use sans-serif fonts otherwise (or if you simply prefer 
them)}}
  \end{itemize}
\item ... and not so good ones:
  \begin{itemize}
  \item {\texttt{Never use monospace for normal text}}
  \item {\frakfamily Gothic, calligraphic or weird fonts: should always: be
  avoided}
\end{itemize}
\end{itemize}
\end{frame}

\begin{frame}[fragile]{Look}
\begin{itemize}
\item To insert a final slide with the title and final thanks, use \verb|\backmatter|.
      \begin{itemize}
      \item The title also appears in footlines along with the author name, you can change this text with \verb|\footlinepayoff|
      \item You can remove the title from the final slide with \verb|\backmatter[notitle]|
      \end{itemize}
\item The aspect ratio defaults to 16:9, and you should not change it to 4:3
      for old projectors as it is inherently impossible to perfectly convert a 
      16:9 presentation to 4:3 one; spacings \emph{will} break
      \begin{itemize}
      \item The \texttt{aspectratio} argument to the \texttt{beamer} class is
            overridden by the SINTEF theme
      \item If you \emph{really} know what you are doing, check the package
            code and look for the \texttt{geometry} class.
      \end{itemize}
\end{itemize}
\end{frame}

\begin{frame}{Citation}
\begin{itemize}
\item you can cite your reference use \texttt{$\backslash$cite$\{\}$}, e.g. \cite{bagla2005cosmological}. The Reference will be shown in the last page.
\end{itemize}
\end{frame}

\section{Summary}

\begin{frame}
\frametitle{Good Luck!}
\begin{itemize}
\item Enough for an introduction! You should know enough by now
\item If you have corrections or suggestions,
\hrefcol{mailto:1155168718@link.cuhk.edu.hk}{send them to me!}
\end{itemize}
\end{frame}

\section{Reference}

\begin{frame}[allowframebreaks]{References}
\tiny
\bibliographystyle{apalike}
\bibliography{bibliography}
\end{frame}

\backmatter
\end{document}
